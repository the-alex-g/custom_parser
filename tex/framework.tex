\documentclass[letterpaper, 12pt, twocolumn]{book}
\usepackage{ragged2e}
\usepackage[left=0.5in, right=0.5in, top=1in, bottom=1in]{geometry}
\usepackage{graphicx}
\usepackage{tabularx}
\def\halfline{\makebox[\columnwidth]{\rule{3.7in}{0.4pt}}\\}
\def\unnumberedsection#1{\section*{#1}\addcontentsline{toc}{section}{#1}}
\def\unnumberedsubsection#1{\subsection*{#1}\addcontentsline{toc}{subsection}{#1}}
\def\unnumberedsubsubsection#1{\subsubsection*{#1}\addcontentsline{toc}{subsubsection}{#1}}
\def\example#1{\textit{Example:} #1}

\def\spell#1#2#3#4{\textbf{#1 (#2)}

\textit{Duration: #3}

#4\bigskip}
\def\rangespellattack#1#2#3{Make a d#1 #2 spell attack (range #3).}
%bound extradimensional entity that is trying to escape
\def\bind#1#2{Summon any #1 that you know the summoning ritual for. If the creature has one or fewer health, you may summon three of them.
The #1 appears nearby and obeys your commands unquestioningly. Whenever you take damage or cast another spell, you must make a Det check
againt the summoned creature. On a failure, the creature breaks free from your control and acts independently until the spell ends. When
the spell ends or the summoned creature is killed, the creature #2.}
%summoned living creature
\def\summon#1#2#3{Summon any #1 that you have seen before, and that has fewer than #2 health. The #1 appears nearby and obeys simple
commands that are not obviously suicidal. If the creature is still alive when the spell ends, it #3.}
%created creature
\def\create#1#2{Summon any #1 that you know the summoning ritual for. If the creature has one or fewer health, you may summon three of them.
The #1 appears nearby and obeys your commands unquestioningly. When the spell ends or the summoned creature is killed, the creature #2.}
\def\spelldrain#1{In order to cast this spell, you must suffer #1 of spell drain.

}

% comprehensive armor listing (from low to high, || indicates base armor values):
% d12/11, d10/9, d8/7, |d6/5|, d10/8, |d4/3|, d10/7, |d6/4|, d8/5, d10/6, d12/7, |d6/3|, d12/5
% d10/4, d8/3, |d6/2|, d10/3, |d4/1|, d10/2, |d6/1|, d8/1, d10/1, d12/1

% armor    dice  health mod
% type 0 = d6/5 = x6/5
% type 1 = d4/3 = x4/3
% type 2 = d6/4 = x3/2
% type 3 = d6/3 = x2
% type 4 = d6/2 = x3
% type 5 = d4/1 = x4
% type 6 = d6/1 = x6

\begin{document}
\RaggedRight

\tableofcontents
\markboth{}{}

\unnumberedsection{Rules}

\unnumberedsubsection{Stuff}

Your Evasion is equal to your Speed bonus, plus one if your Dex bonus is +2 or higher, or -1 if your Dex bonus is -2 or lower, minus your
Size bonus.
Your Health is equal to 4 plus your Con bonus.

\unnumberedsubsection{Attacking}

When you make an attack, you must first choose an equipped weapon to attack with and a target within range of that weapon.
Roll that weapon's attack die. If the result is a 1, or less than or equal to the target's Evasion minus your Dexterity, the attack misses.
Otherwise, the attack hits, inflicting one wound on the target. Furthermore, if the result is more than 4 and also greater than the size
of the die minus your Strength, or if the result is the maximum possible value, you may make another attack against the same target.

\example{Fredrick wants to attack a goblin using his longsword. Fredrick has +2 Strength and +0 Dexterity, and the goblin has
an Evasion of 2. The following table shows the possible results for Fredrick:
\begin{tabularx}{\columnwidth}{|c|X|}
    \hline
    d8 & Result \\
    \hline
    1-2 & Miss, because the roll is less than or equal to the target's Evasion. \\
    \hline
    3-6 & Hit for one wound. \\
    \hline
    7-8 & Hit for one wound and roll again, because the value is greater than the size of the die minus Fredrick's strength (6). \\
    \hline
\end{tabularx}
}

\unnumberedsubsubsection{Defending}

If a monster is attacking a player character, resolve as above. After the attack has been rolled and the number of incoming wounds is
known, the player rolls their Armor die once for each wound. For each result equal to or greater than the armor's threshold, one wound
is negated.

\example{The goblin attacks Fredrick back, hitting for 1 wound. Fredrick has type 4 armor (d6/4), so rolls a d6 once. If he rolls a 
4 or better, the wound is negated.}


\unnumberedsubsection{Light and Sensing}

\begin{tabularx}{\columnwidth}{|c|X|}
    \hline
    Bright Light & No penalties \\
    \hline
    Dim Light & +2 Evasion \\
    \hline
    Darkness & Creatures in darkness cannot be seen \\
    \hline
    Utter Darkness & Creatures in utter darkness cannot be seen \\
    \hline
    \hline
    Blindsight & Can sense surroundings, regardless of ability to see \\
    % for creatures that cannot see with eyes, but can still detect their surroundings
    \hline
    Darksight & Can see normally, regardless of light level \\
    % magical in nature
    \hline
    Dimsight & Can see as if the light level were one higher \\
    % natural keen night vision
    \hline
    Sense Living & Can sense living creatures, regardless of ability to see \\
    % for telepaths and powerful undead
    \hline
    Tremorsense & Can sense moving things, regardless of ability to see \\
    % for things like plants without another way to sense and burrowing monsters
    \hline
    Truesight & Can see perfectly regardless of light level, can see invisible creatures and objects, and can immediately identify visual illusions \\
    % extremely magical in nature
    \hline
\end{tabularx}


\unnumberedsubsection{Spellcasting}

When you cast a spell, unless its duration is instantaneous or permanent, roll 1d4 and add your spellcasting modifier. The result is how
many units of time the spell lasts.
\example{Althier, a wizard, casts \textit{Shield}, with a duration of rounds. He rolls 1d4 and adds his Lor bonus, getting a total of
3. Therefore, his shield spell lasts for three rounds.}

If the spell directly targets a creature with spell resistance, the cost of the spell is increased by that creature's spell resistance.
If not directly targeted by a spell, but still affected by one (like being in the radius of a \textit{Fireball}), that creature adds its
spell resistance to any checks required by the spell.

\unnumberedsubsubsection{Clerics}

A cleric's spellcasting ability is either Insight (usually for good deities), Determination (usually for evil deities), or Charisma.

When you cast a spell, you must make a check with your spellcasting ability. The difficulty is equal to 20 - your favor points + twice
the level of the spell. If you fail, you lose one favor point and the spell fails. Each day, you gain one favor point if you have
fewer than 8, and you lose one favor point if you have more than 10. You may never have fewer than 0 or more than 16 favor points.

For every hour you spend in prayer to your deity, you gain one favor point. You cannot regain more than three favor points in this
way per day.

\textit{Spell Drain:} For each point of spell drain, you lose one favor point.


\unnumberedsubsubsection{Druids}

A druid's spellcasting ability is Insight.

Druids can cast six level 1 spells, three level 2 spells, and one level 4 spell. While in an area filled with natural life,
an additional two level 1 spells and one level 2 spell can be cast. At dawn each day, you regain all your extra spells for
casting in natural surroundings, as well as either two level 1 spells, one level 2 spell, or one level 4 spell.

\textit{Spell Drain:} For each point of spell drain, you count as having cast one spell of the lowest level that you can still cast.
\example{Dryxi has cast five level 1 spells, and then suffers an effect inflicting two points of spell drain. Dryxi loses his remaining
level 1 spell and one of his remaining level 2 spells.}


\unnumberedsubsubsection{Shamans}

A shaman's spellcasting ability is Determination.

When you cast a spell, you must make a Det 14 check. If you fail, you take unblockable light wounds equal to the level of the spell.
Regardless of success or failure, the spell is cast.

\textit{Spell Drain:} Suffer one unblockable light wound for each point of spell drain.


\unnumberedsubsubsection{Wizards}

A wizard's spellcasting ability is Lore.

Wizards have a pool of 16 mana points. When you cast a spell, you must spend a number of mana points equal to the spell level. If you don't
have enough mana points to pay for the spell, you cannot cast it.

As an action, you may make a Lor 14 check. If you succeed, you regain 1d3 mana points. If you fail, you take 1d3 unblockable light wounds.
Each day that you get a full night's rest, you regain mana points equal to your spellcasting modifier (minimum of 1).

\textit{Spell Drain:} Your maximum number of mana points is reduced by the amount of drain. Each day that you get a full night's rest, your
maximum number of mana points is increased by one, to a maximum of the value before drain was applied.

\unnumberedsection{Spells}

%[spells]

\unnumberedsection{Bards}
Bards have a number of saga points equal to their Charisma bonus plus the number of languages they know. As an action, you may start
any song you know. Whenever you take damage, or take an action other than continuing the song, you must succeed on a Cha 12 check or
the effects of the song end. As an action, you may end one song currently in effect and spend a saga point to activate that song's
climax effect.

%[themes]

\unnumberedsection{Monsters}

%[monsters]

%[appendicies]

\end{document}