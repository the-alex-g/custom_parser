\documentclass[letterpaper, 12pt, twocolumn]{book}
\usepackage{ragged2e}
\usepackage[left=0.5in, right=0.5in, top=1in, bottom=1in]{geometry}
\usepackage{graphicx}
\usepackage{tabularx}
\def\halfline{\makebox[\columnwidth]{\rule{3.7in}{0.4pt}}\\}
\def\unnumberedsection#1{\section*{#1}\addcontentsline{toc}{section}{#1}}
\def\unnumberedsubsection#1{\subsection*{#1}\addcontentsline{toc}{subsection}{#1}}
\def\unnumberedsubsubsection#1{\subsubsection*{#1}\addcontentsline{toc}{subsubsection}{#1}}
\def\example#1{\textit{Example:} #1}
\begin{document}
\RaggedRight

\unnumberedsection{Rules}

\unnumberedsubsection{Attacking}

When you make an attack, you must first choose an equipped weapon to attack with and a target within range of that weapon.
Roll that weapon's attack die. If the result is a 1, or less than or equal to the target's Evasion minus your Dexterity, the attack misses.
Otherwise, the attack hits, inflicting one wound on the target. Furthermore, if the result is more than 4 and also greater than the size
of the die minus your Strength, or if the result is the maximum possible value, you may make another attack against the same target.

\example{Fredrick wants to attack a goblin using his longsword. Fredrick has +2 Strength and +0 Dexterity, and the goblin has
an Evasion of 2. The following table shows the possible results for Fredrick:
\begin{tabularx}{\columnwidth}{|c|X|}
    \hline
    d8 & Result \\
    \hline
    1-2 & Miss, because the roll is less than or equal to the target's Evasion. \\
    \hline
    3-6 & Hit for one wound. \\
    \hline
    7-8 & Hit for one wound and roll again, because the value is greater than the size of the die minus Fredrick's strength (6). \\
    \hline
\end{tabularx}
}

\unnumberedsubsubsection{Defending}

If a monster is attacking a player character, resolve as above. After the attack has been rolled and the number of incoming wounds is
known, the player rolls their Armor die once for each wound. For each result equal to or greater than the armor's threshold, one wound
is negated.

\example{The goblin attacks Fredrick back, hitting for 1 wound. Fredrick has type 4 armor (d6/4), so rolls a d6 once. If he rolls a 
4 or better, the wound is negated.}

%[monsters]

\end{document}